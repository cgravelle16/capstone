\section{Final Design}
The final design of the robots drive system is very similar to the proposed design provided to Penguin ASI in December. Using a belt drive over a chain drive was accepted through our proposal along with keeping it battery powered. In the final design, four batteries that Penguin ASI insisted we used were still required to power all for brushless DC motors and two other batteries required to supply power to the other electronic components on board the robot. The final design also had features that were in the proposed design, such as removable interior components, easy serviceability and keeping it as a modular design. Having the components easily removable will allow anyone who is maintaining the drive system have access to all the bearings for re-greasing and easy vision for inspection of belts condition.  After assembling one exterior drive box, some small design changes were made that were not noticed in the CAD model assemblies or were notified to us from Penguin Employees after the proposal report had been submitted. None of these changes required the proposed design to drastically change but actually improved the overall design and reduced the amount of machining time required for some components. An example of reducing machining time would be the new hubs located at the back side do the drive boxes. In the proposed design the rear hubs required machining on both the exterior and interior, but with the new design all exterior machining was not required.Below is a table with the robots specifications.


