\section{Project Management}

From the beginning, project management was crucial to keeping our team working efficiently and on schedule in order to complete the proposed final design. Do to the structure of the course, we added two main milestones: the design phase and build phase. The first half of the semester was dedicated to designing the solution to the problem. During this time, a work schedule was created using WBS (Work Breakdown Structure) to get a better understanding of the scope of the project and division of tasks. Afterwards, a Gantt chart was produced. Both the the WBS and Gantt chart underwent many revisions to complete the project in the allocated amount of time. The final WBS is in Figure~\ref{fig:wbs} in Appendix~\ref{wbs}.

Milestones were set at many points throughout the project to ensure that all work stayed on schedule. The first major milestone was to get one drive box completed to ensure that all designed components would fit together and interact as desired before creating the other drive boxes. The manufacturing schedule during this phase was developed to accommodate lead times. For instance, parts not needing exact measurements from other purchased components were produced first. After the completion of the first drive box, time was allotted to implement minor changes for ease of manufacturing. Then began work on the stretch goals: completing the other 3 drive boxes. Also, given that the robot frame was to be reused, modifications needed to be made to it to accommodate the new design. The frame modifications were done in parallel with machining of drive box components, as well as testing the controller.  This ended up being one of the most times consuming areas of the building phase.

Critical documentation deadlines were also set as milestones. Reports and presentations were typically done nearing the end of the every phase. This had as an advantage to include all current changes before submission of the documents and to have all necessary information to complete the report at once. 

The number of hours needed to complete the project was staggering, hence the need for such a rigorous schedule. An initial estimate of the time needed to complete the project was 182 hours on the CNC Lathe, 146 on the CNC mill and 61 hours on frame accommodation. Thus, each team member needed to spend at a minimum 30 hours a week on manufacturing alone. As such, there was need to coordinate with Penguin and Laurentian University to reserve shop time; Tuesdays, Wednesdays, and Saturdays were set aside to work on the project. In total, all member of the team worked approximately 270 hours in manufacturing which is reflected in the Gantt chart. The final Gantt chart can be found~\ref{fig:gantt} in Appendix~\ref{gantt}. Moreover, to increase efficiency, an ``assembly line'' like procedure was developed during the second phase of building to ensure that all resources, team members and machines, were used to their full capacity.