\subsection{Pivot and Travel Limiter}
\subsubsection{Description}

The minimum required shaft diameter was calculated using two iterations of DE Goodman shaft design theory. The second iteration took into account more realistic stress concentrations allowing for the reduction in required shaft diameter. The shaft layout and loading assumptions are illustrated in Figure 24 forces at the ends representing the radial forces applied from the taper roller bearings and the force at the middle being the resultant force of the sprocket determined from the Gates Design IQ software. Bearing calculations were performed according to the procedure outlined in the SKF bearing design manual for taper roller bearings. It is important to note that the free body diagram in Figure 24 only represents the radial component of the force from the taper roller bearing, and axial forces not shown were also taken into consideration. Design constraints and requirements are outlined in Table 18.  

\subsubsection{Design Constraints}

\subsubsection{Functional Requirements}

\subsubsection{Alternate Solution}

\subsubsection{Analysis and Design}

The finite element analysis report for all three revisions of the output shaft can be found in Appendix A.  The revisions 0 and 2 can be seen in the Figure 25 and Figure 26 below. In revision 0 the stresses experienced at the end of the output shaft actually exceed the yield strength of the 4340 steel. The changes between the two shaft layouts are significantly different since the first revision was designed to mount radial bearings onto and the final revision now has a taper roller bearing on each end of the sprocket. Seals are also located on both sides of the taper roller bearing to provide a seal. The stresses experienced in the final revision are greatly reduced and do not exceed the yield strength of the material selected. Fillets were also added to reduce the stress concentrations at the shoulders of the different diameters on the shaft.
