\usepackage{gensymbol}
\section {Design Alternatives}
The main feature to the project was based off of power transmission from the motors to the wheels. For this, three different methods were explored. The first consideration was a chain drive which was also found on the previous iteration. However, chains proved to required quite a bit of maintenance and must operate in a sealed and lubricated gearbox. Aside from sealing concerns, corrosion and stretching from use were other issues that posed problems and increased maintenance time. 

The second was a serial shaft configuration that would require many 90\degree joints which are expensive, bulky and high maintenance as well. This last options was nearly eliminated right off the hop but was kept for comparison and analysis in the decision making process. 

The third and final option was a synchronous belt drive system. After some research and discussions with the engineers at Gates, one of the world's leading manufacturers in belt systems, it was concluded that a belt drive would be very suitable for the application at hand. Operating in a dry environment, virtually no stretch over it's lifetime and a significant increase in longevity over a chain drive are all features that made belts much more attractive. As claimed by Gates, belts are to last 3 times longer than chains and sprockets for belts are to last 10 times longer than ones for chains as claimed by gates. It is to note that belt don't take side loading or twisting very well, therefore much emphasis on reducing these affects were included in the design of the drivebox structure. 

Belts were chosen after careful evaluation against the other alternatives as seen in the Pugh matrix in Appendix ~\REF~{pugh}





