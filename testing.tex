\section{Testing}
\subsection{Initial Build}
While the first drive box was being manufactured, the code for the controller was tested with the motors and the batteries as a dry run to ensure that the program functioned as required, with a top speed of 3 km/h. The controller was tested by adjusting the voltage that the controller provided to the motors and both direction of rotation. After the completion of one assembled drive box it was placed into the chassis of the robot with the motor and the controller. The assembly was then tested without wheels or loading to ensure that all components within the drive box worked together and that no rubbing or undesired out comes were to happen. During this testing all components operated correctly and was able to handle the maximum rotation to achieve the desired speed of 3 km/h. During the manufacturing of the initial drive box some design changes were brought up that would help both improve our design along with decrease the manufacturing time of some components. The 1/4" aluminum plate that was required on the back plate was dropped and the hole was filled on the first 1" plate to enclose the drive box. The rear hubs were also changed to ensure that sealing would be achieved at the back. The rear hubs design also changed to reduce machining time and to make the assembly easier for inserting both the shaft seals and bearing races. A flexible Lovejoy coupling was implemented instead of the custom one that was proposed to allow mounting of the output shaft to the reduction shaft a simpler task during assembly and to help with any misalignment that may have occurred between the two during the frame modifications. The cut outs on the 1" rear and front drive boxes were changed to decrease the stress concentrations at the corners and to reduce the amount of machining time for each plate. For pivot the numbers of fasteners used to mount to the drive box was reduced from 12 bolts to 10 since it was unnecessary to have that many and it decreased the amount of fasteners used in the assembly, while reducing machining time. A snap ring grove was added to the output shaft to prevent the coupling from sliding back on the shaft and losing the connection to the reduction shaft.

\subsection{Final Build}
When the final build was completed it was tested using the same program and controller as the initial test and the assembly worked accordingly. The improvements made allow us to cut the amount of machining time used and allowed for an easier assembly while keeping the same desirable outcomes of the design. In the final testing done by us the controller was connected to two drive boxes and running them under no loads to ensure that but operated correctly together. The controller and the assemblies of the drive boxes work as designed for both the forward and reverse directions of rotation.

\subsection{FMEA}
