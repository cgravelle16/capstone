\section{Loading Conditions}
As with any off road vehicle, it is always difficult to predict loading conditions as there is an infinite amount of scenarios due to the varying environment and terrain, however it can usually be generalized to a few simple, yet appropriate cases. Also, it is important for one to remind themselves what the vehicle is being designed for. Although it is an off-road vehicle, it is not designed for extreme conditions or dangerous scenarios. It's the same logic for someone who owns an economical car; they wouldn't be taking this vehicle into the bush through deep mud or rough terrain as it will most likely end up with broken parts. In this case, the robot is to be operated in a normal mining environment where the environment conditions are assumed to be a regular mining road built from crushed, roughly fist sized rock, and a max ramp grade of 20\% or incline of 11.3 degrees. Driving forward up a ramp and a skid steer turn on a ramp were then evaluated. As one could imagine, the skid steer motion when the robot is balancing on two, opposite diagonal wheels resulted in the highest forces, therefore these were used for analysis. With the weight distribution of the coupled wheels being 40/60, the 60\% wheel is used for analysis and its design is to be duplicated to the other wheel. The resulting forces are summarized in     
\line
 ***INSERT TABLE***
\line
It should be noted that only a static analysis was conducted for this project due to slow operating speeds. With a top speed of 3 km/h, the forces due to dynamic loading would only differentiate slightly. To compensate for this, a larger factor of safety was desired for each component. Given an adequate time line, one could conduct a full dynamic simulation to obtain more accurate results and confirm that the design to follow is indeed adequate for the imposed loading conditions. 

\section{Belt Drive}
\subsection{Description}
\subsection{Design Constraints}
\subsection{Functional Requirements}
\subsection{Alternate Solutions}
\subsection{Analysis and Design}

\section{Wheel Shaft Assembly}
\subsection{Description}
\subsection{Design Constraints}
\subsection{Functional Requirements}
\subsection{Alternate Solutions}
\subsection{Analysis and Design}
\subsubsection{Wheel Shaft}
\subsubsection{Wheel Bearings}
\subsubsection{Shaft Seals}
\subsubsection{Front Bearing Housings}
\subsubsection{Rear Bearing Housing}
\subsubsection{Rim Mount}