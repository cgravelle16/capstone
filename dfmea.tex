\section{DFMEA} \label{dfmea}

\begin{landscape}
\begin{table}
\caption[DFMEA]{Design Failure Modes and Effect Analysis for Drive Box}
\footnotesize
\begin{tabular}{| p{1.5cm}|p{3cm}p{3cm}p{0.5cm}p{3cm}p{0.5cm}p{3.5cm}p{0.5cm}p{0.5cm}p{3cm} |} \hline
Purpose & Potential Failure Mode & Potential Effects of Failure & SEV & Potential Cause of Failure & OCC & Current Controls Evaluation Method & DET & RPN & Recommended Actions \\ \hline
\multirow{2}{1.5cm}{Seal} & Plates bend and allow space between o-ring and plate & Premature wear to belts/reduce belt life & 7 & Plate cannot support robot weight & 4 & Water inside drivebox, o-rings cut/damaged, belt slippage & 9 & 252 & Reinforce plates, check bolt tension \\
& Seals fail & Cause belt slip  & 8 & Deflection in drive box components & 4 & Water inside drivebox, belt slippage & 6 & 192 & Reinforce plates, check bolt tension \\ \hline
\multirow{5}{1.5cm}{Transmit Torque} & Belt breaks & Vehicle doesn't move & 9 & Sudden tension increase & 3 & Wheels don't turn, output from vehicle is reduced & 3 & 81 & Replace the belt  \\
 & Belt slips & Vehicle partially moves & 8 & Seals fail & 3 & Output from vehicle is reduced, not as responsive, noise & 6 & 144 & Revise sealing issue \\
 & Bearings too tight and causes too much friction & Premature wear of bearings & 6 & Improper fitting of bearing & 2 & Wheels don't turn freely & 8 & 96 & Machine to proper tolerance \\
 & Bearings break & Wheels stop rolling, vehicle skids. Damage to belts & 7 & Improper fitting of bearing & 1 & Unwanted noise & 3 & 21 & Replace bearing and observe cause of breaking \\
 & Back plate deforms & Inoperable vehicle & 10 & Material strength too low & 6 & Angled drivebox & 7 & 420 & Reinforce back plate \\ \hline
\multirow{5}{1.5cm}{Support Robot Weight} & Pivot pierces through back plate & Dysfunctional drivebox & 9 & Plate not thick enough, too much force & 5 & Dragging drivebox, drivebox falls off & 1 & 45 & Redesign back plate \\
 & Plate deflection & Belt snaps, possible break of seal - vehicle cannot drive & 5 & Plate not thick enough too much force & 3 & Snapped belt, visual/angled drivebox & 6 & 90 & Increase radii of drive box plate cutouts \\
 & Wheel shafts deflect (significant) & Added stress on bearings & 3 & Too much weight on shaft & 1 & Angled wheels & 2 & 6 & Revise shaft design \\
 & Bushing failure & Drive box dismounts from frame & 7 & Bushing can't handle skid steer & 3 & Play in bushing, drivebox doesn't spin freely, noise & 1 & 21 & Replace bushings or create new thicker bushings \\
 & Fasteners fail at pivot connection of drive box & Drive box dismounts from pivot & 7 & Skid steer forces to greater than estimated & 3 & Dragging drivebox, drivebox falls off & 1 & 21 & Use a higher grade of fastener for increase strength \\ \hline
 \end{tabular}
 \label{tab:dfmea}
 \end{table}
 \end{landscape}
